\documentclass[12pt]{article}
\usepackage{amsmath}
\usepackage{framed, enumitem}

\title{Automobile Manufacturing Report}
\author{Eddie Solares}
\date{}


\begin{document}
\maketitle

\begin{framed} 
\textbf{Identify possible causes.} From experience, what could have caused this deviation from what has worked before to what is currently happening? 
\end{framed}

From experience from the past year, the cause of the deviation was the use of silicon based hand cream being used among employees. This hand cream prevented proper paint adhesion in the past. In our current situation, visual inspectors have rejected a significant of EX25 models due to gaps in paint coverage that appear randomly on the surface of the body. Experience alone tells us that the use of silicon hand cream being used among workers during the manufacturing process could be the main cause. From the interview with the quality control supervisor, his inspectors suspect that innapropriate hand cream is being utilized once again. Although experience tells us that this could be the cause, there could be other possible causes from innapropriate hand creams to contaminants leaking in the process. 

\begin{framed}
\textbf{Evaluate possible causes.} How will we evaluate the possible causes proposed? Which possible cause best fits the known facts. Which possible cause has the fewest,
simplest and most reasonable assumptions?
\end{framed}

The cause which has the fewest, simplest, and most reasonable assumptions are that employees are using innapropriate hand lotion. In the past, it was silicon based hand lotion, however, we should make sure to generalize this to innapropriate hand lotion not just based on silicon. When the notice was posted about the acceptable hand creams in the washroom the reject rates returned to normal, so it is a priority to put up these notices once again for the new workers and the old workers who have forgotten.

\begin{framed}
\textbf{Confirm the true cause.} What can we do to verify any assumptions made. How can this cause be observed at work? How can we demonstrate the cause-and-effect relationship? When corrective action is taken, how will results be checked?
\end{framed}

Now let us asses what we know about the situation. We know that an increase from $1.5\%$ to $6\%$ in the rejection rate has only occurred among the EX25 models and not EX35 and EX45 models which are both at the average of $1.5\%$. Since the EX25 model is produced at a higher rate than the EX35 and EX45 model they are on different assembly lines. So the focus is on the first assembly line which contains the EX25. Also, since the central supply tank supplies both production lines, and the second production line is not producing paint defects we can focus on the individual wash tanks in the first assembly line. 

In the first assembly lines there may be a number of places where contaminants or the use of the improper hand lotion could have occured. First we can check the cleaning solution to see if anyone is handling that or if the cleaning solution is not properly emptied and cleaned out. Also, when the body is manually removed and carried over to the hangar we could have possible innapropriate lotion or contaminants touching the body. Finally, if anyone is handling the small batches with innapropriate hand cream or contaminants in the homogenizer it may affect the painting. These all would be places to take a closer look at and carefully moniter. 

Now when these corrective actions are taken our results can be simply checked by testing the rejection rate during the following weeks. We want to make sure that the rejection rates for the EX25 model drop from $6\%$ back down to the normal $1.5\%$. Since the rates of production are rising due to high demand, it is important that we take the rate relative to the current production, and not compare to the previous. 



\end{document}